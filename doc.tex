% !TeX root = doc.tex
\documentclass[a4paper,12pt]{book} % nie: report!


\usepackage[T1,plmath]{polski}\usepackage{polski} % lepiej to zamiast babel!
\usepackage[utf8]{inputenc} % w razie kłopotów spróbować: \usepackage[utf8x]{inputenc}
\usepackage{fancyhdr} % nagłówki i stopki
\usepackage{indentfirst} % WAŻNE, MA BYĆ!
\usepackage[pdftex]{graphicx} % to do wstawiania rysunków
\usepackage{amsfonts}
\usepackage{amsmath} % to do dodatkowych symboli, przydatne
\usepackage[pdftex,
            left=1in,right=1in,
            top=1in,bottom=1in]{geometry} % marginsy
\usepackage{amssymb} % to też do dodatkowych symboli, też przydatne
\usepackage{amsthm}
\usepackage{float}
\usepackage[font=small,labelfont=bf]{caption}
% jesli potrzeb, można oczywiście wstawić inne pakiety i swoje definicje...

\usepackage{amsmath}





% definicje nagłówków i stopek
\pagestyle{fancy}
\renewcommand{\chaptermark}[1]{\markboth{#1}{}}
\renewcommand{\sectionmark}[1]{\markright{\thesection\ #1}}
\fancyhf{}
\fancyhead[LE,RO]{\footnotesize\bfseries\thepage}
\fancyhead[LO]{\footnotesize\rightmark}
\fancyhead[RE]{\footnotesize\leftmark}
\renewcommand{\headrulewidth}{0.5pt}
\renewcommand{\footrulewidth}{0pt}
\addtolength{\headheight}{1.5pt}
\fancypagestyle{plain}{\fancyhead{}\cfoot{\footnotesize\bfseries\thepage}\renewcommand{\headrulewidth}{0pt}}

% interlinia
\linespread{1.25}






\begin{document}
\begin{titlepage}
  ~

  \begin{tabular}{c@{\hspace{21mm}}|@{\hspace{5mm}}l}
    \vspace{-20mm} &                                                                   \\
    \multicolumn{2}{l}{\hspace{-12.5mm} \includegraphics[width=8cm]{UMCS_podstaw_12G_PL_P431.jpg}}     \\
    \multicolumn{2}{@{\hspace{20mm}}l}{\vspace{-4mm}}                                  \\
    \multicolumn{2}{@{\hspace{28mm}}l}{\Large \sf UNIWERSYTET MARII
    CURIE-SKŁODOWSKIEJ}                                                                \\
    \multicolumn{2}{@{\hspace{28mm}}l}{\vspace{-4mm}}                                  \\
    \multicolumn{2}{@{\hspace{28mm}}l}{\Large \sf W LUBLINIE}                          \\
    \multicolumn{2}{@{\hspace{28mm}}l}{\vspace{-4mm}}                                  \\
    \multicolumn{2}{@{\hspace{28mm}}l}{\Large \sf Wydział Matematyki, Fizyki i
    Informatyki}                                                                       \\
    \multicolumn{2}{@{\hspace{28mm}}l}{\vspace{21mm}}                                  \\
                   & {\sf Kierunek: \textbf{informatyka} }                             \\
    %& {\sf Specjalność: \textbf{informatyczna}} \\ % wpisujemy tylko jeśli jest!!!
                   &                                                                   \\\\\\
                   & {\sf \large \bfseries Patryk Wałach}                              \\
                   & {\sf nr albumu: 296597}                                           \\
                   &                                                                   \\\\\\
                   & \Large \sf \bfseries Tytuł pracy    \\
                   & \Large \sf \bfseries ...        \\
                   & \Large \sf \bfseries ...                         \\\\[-10pt]
                   & {\large \sf English title}         \\
                   & {\large \sf ...}                \\
                   & {\large \sf ...}                               \\
                   &                                                                   \\
                   &                                                                   \\
                   &                                                                   \\
                   & {\sf Praca magisterska}                                           \\
                   & \vspace{-7mm}                                                     \\
                   & {\sf napisana w Katedrze Oprogramowania Systemów Informatycznych} \\
                   & {\sf Instytutu Informatyki UMCS}                                  \\
                   & \vspace{-7mm}                                                     \\
                   & {\sf pod kierunkiem \bfseries dr hab. Jarosława Byliny}           \\
    \multicolumn{2}{@{\hspace{28mm}}l}{\vspace{15mm}}                                  \\
    \multicolumn{2}{@{\hspace{28mm}}l}{\textbf{\textsf{Lublin 2024}}}
  \end{tabular}
\end{titlepage}





\sloppy



\thispagestyle{empty}


\newpage{}

\thispagestyle{empty}

\newpage{}



\tableofcontents{}






\chapter*{Wstęp}
\addcontentsline{toc}{chapter}{Wstęp} % ...ale w spisie treści ma być
% napisać na sam koniec

\chapter{Ogólne wprowadzenie}
\section{Kaskadowe arkusze stylów (ang. \emph{Cascading Style Sheets}, w skrócie CSS)}
\section{Konwencja Blok Element Modyfikator (w skrócie BEM)}
\section{Wszechstronny CSS (ang. \emph{utility-first CSS})}
\subsection{statyczny}
\subsection{dynamiczny}

\section{CSS w JavaScript}

\chapter{Problemy języka CSS}
\section{rosnące rozmiary plików}
\section{kolizje pomiędzy stylami}
\section{rozmieszczanie klas pomiędzy plikami css}

\chapter{Najczęściej wykorzystywane funkcje języka CSS}
\section{selektory}
\section{pseudo-selektory}
\section{nadpisywanie stylów}
\section{dziedziczenie stylów}
\section{style warunkowe (media query)}
\section{style dynamiczne}


\chapter{Narzędzia}
\section{Język JavaScript}
\subsection{Środowisko Node.js}
\section{React}
\subsection{Nextjs}
\section{Storybook}
\subsection{Chromatic}


\chapter*{Podsumowanie}
\addcontentsline{toc}{chapter}{Podsumowanie}

% napisać na sam koniec

\bibliographystyle{abbrv}

\bibliography{doc}
\addcontentsline{toc}{chapter}{Bibliografia}

\end{document}